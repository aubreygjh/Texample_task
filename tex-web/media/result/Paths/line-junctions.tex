% Block diagram wire junctions
\documentclass{article}
\usepackage{tikz}
\usetikzlibrary{arrows}
\usepackage{verbatim}

\begin{comment}
:Title: Block diagram line junctions
:Slug: line-junctions
:Tags: Block diagrams, Foreach, Transformations, Paths

An example of how to draw line junctions in a block diagram. 
A semicircle is used to indicate that two lines are not connected. 
This is a good example of how flexible TikZ' paths are. 
The intersection between the lines are calculated using the convenient 
``-|`` syntax. Since we want the semicircle to have its center where 
the lines intersect, we have to shift the intersection coordinate 
accordingly. 

\end{comment}


\begin{document}

\tikzstyle{block} = [draw,fill=blue!20,minimum size=2em]
% diameter of semicircle used to indicate that two lines are not connected
\def\radius{.7mm} 
\tikzstyle{branch}=[fill,shape=circle,minimum size=3pt,inner sep=0pt]

\begin{tikzpicture}[>=latex']

    % Draw blocks, inputs and outputs
    \foreach \y in {1,2,3,4,5} {
        \node at (0,-\y) (input\y) {$i_\y$};
        \node[block] at (2,-\y) (block\y) {$f_\y$};
        \draw[->] (input\y) -- (block\y);
        \draw[->] (block\y.east) -- +(0.5,0);
    }
    \node[block] at (2,-6) (block6) {$f_6$};
    \draw[->] (block6.east) -- +(0.5,0);

    % Calculate branch point coordinate
    \path (input1) -- coordinate (branch) (block1);

    % Define a style for shifting a coordinate upwards
    % Note the curly brackets around the coordinate.
    \tikzstyle{s}=[shift={(0mm,\radius)}]
    % It would be natural to use the yshift or xshift option, but that does
    % not seem to work when shifting coordinates.

    \draw[->] (branch) node[branch] {}{ % draw branch junction
            \foreach \c in {2,3,4,5} {
                % Draw semicircle junction to indicate that the lines are
                % not connected. The intersection between the lines are
                % calculated using the convenient -| syntax. Since we want
                % the semicircle to have its center where the lines intersect,
                % we have to shift the intersection coordinate using the 's'
                % style to account for this.
                [shift only] -- ([s]input\c -| branch) arc(90:-90:\radius)
                % Note the use of the [shift only] option. It is not necessary,
                % but I have used it to ensure that the semicircles have the
                % same size regardless of scaling.
            }
        } |- (block6);
\end{tikzpicture}
\end{document}