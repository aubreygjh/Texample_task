% Rooty helix
% Author: Felix Lindemann
\documentclass{minimal}

\usepackage{tikz}
\usetikzlibrary{calc}
\usepackage{verbatim}

\begin{comment}

:Title: Rooty helix

The idea of the rooty-helix is very simple. One starts e.g. with the length of 1, adds a right angle with the length of 1 and the hypotenuse equals \sqrt{2}. If one continues with \sqrt{2} repeating the procedure (adding a right angle with the length of 1) the hypotenuse is \sqrt{3}. And so on and so on. At some point due to the iterations some triangles are overpainted. Because of this, one repaints the overpainted triangles
\end{comment}

\begin{document}

\pagestyle{empty}
\pgfdeclarelayer{background}
\pgfdeclarelayer{foreground}
\pgfsetlayers{background,main,foreground}

\xdefinecolor{darkgreen}{RGB}{175, 193, 36}
\newcounter{cntShader}
\newcounter{cntRoot}
\setcounter{cntShader}{20}
\def\couleur{darkgreen}

\begin{tikzpicture}
    \foreach \y in {86,38,15}{
        \setcounter{cntShader}{1}
        \coordinate (a) at (0,0);
        \coordinate (b) at (0:1);
        \foreach \x in {1,...,\y}{%
            \coordinate (c) at ($ (b)!1cm!270:(a) $);
            \begin{pgfonlayer}{background}
                \draw[fill=\couleur!\thecntShader] (a)--(b)--(c)--cycle;
            \end{pgfonlayer}
            \setcounter{cntRoot}{\x}
            \addtocounter{cntRoot}{1}
            \node[fill=white,draw,circle,inner sep=1pt] at (c)
                {$\sqrt{\thecntRoot}$};
            \coordinate (b) at (c);
            \pgfmathsetcounter{cntShader}{\thecntShader+4}
            \setcounter{cntShader}{\thecntShader}
       }
    }
    \node[fill=white,draw,circle,inner sep=1pt] at (0:1) {$\sqrt{1}$};
\end{tikzpicture}

\end{document} 